\documentclass[11pt]{article}
\usepackage[utf8]{inputenc}
\usepackage[T1]{fontenc}
\usepackage[spanish]{babel}
\usepackage[left=1.5cm, right=1.5cm, top=1.7cm, bottom=1.5cm]{geometry}
\usepackage{hyperref}
\usepackage{enumitem}

\setlength{\parindent}{0pt}
\pagestyle{empty}

\begin{document}

\begin{center}
    {\LARGE \textbf{Salvador Gutiérrez Olvera}} \\
    Teoloyucan, Mex. \hfill Email: salvador.go\_97@hotmail.com \\
    Tel: +52 5582214610 \hfill GitHub: \href{https://github.com/Salvador-97}{Salvador-97}
\end{center}

\vspace{6pt}

\section*{Educación}

\textbf{Ingeniería en Computación (incompleta), Facultad de Ingeniería, UNAM} \hfill Coyoacán, CDMX \\
Agosto 2015 -- Mayo 2019

\textbf{Bachillerato, Colegio de Ciencias y Humanidades Vallejo} \hfill Gustavo A. Madero, CDMX \\
Agosto 2012 -- Junio 2015

\vspace{6pt}

\section*{Experiencia profesional}

\textbf{Almacenista / Embarques} \\
Procesos de Valor Agregado PROVA \hfill Tepotzotlán, Méx. \\
Marzo 2025 -- Septiembre 2025
\begin{itemize}[leftmargin=*, noitemsep]
    \item Registro y control de entradas y salidas de mercancía, asegurando precisión documental.
    \item Conteo y conciliación de inventarios físicos y digitales.
    \item Preparación y surtido de pedidos según requerimientos del sistema.
    \item Registro y análisis de mermas para control de pérdidas y auditorías internas.
    \item Identificación de oportunidades de automatización que dieron origen a proyectos en \textbf{Python} y \textbf{Flask}.
\end{itemize}

\textbf{Ayudante General / Descarga de Contenedores} \\
Almacén Logístico Independiente \hfill Cuautitlán, Méx. \\
Junio 2024 -- Enero 2025
\begin{itemize}[leftmargin=*, noitemsep]
    \item Descarga, clasificación y registro de mercancías.
    \item Control de entradas mediante hojas de registro físico.
    \item Detección de ineficiencias en seguimiento de contenedores, impulsando el desarrollo de un \textbf{sistema web propio}.
\end{itemize}

\vspace{6pt}

\section*{Proyectos de desarrollo}

\textbf{Sistema de generación y gestión de marbetes (Python / Tkinter)} 
\begin{itemize}[leftmargin=*, noitemsep]
    \item Aplicación de escritorio modular para creación automatizada de marbetes y control de productos.
    \item Interfaz gráfica con \textbf{Tkinter} y \textbf{CustomTkinter}, conectada a archivos \textbf{CSV, Excel y JSON}.
    \item Módulos independientes para configuración, edición de personal e información de inventario.
\end{itemize}

\textbf{Sistema de control y registro de contenedores (Flask / Web)} 
\begin{itemize}[leftmargin=*, noitemsep]
    \item Sistema web para registro de contenedores y control de inventario.
    \item Administración de productos, tarimas y ubicaciones mediante \textbf{SQLite}.
    \item Backend con \textbf{Flask (Python)}, frontend con \textbf{HTML, CSS, Bootstrap y JavaScript}.
\end{itemize}

\textbf{Plataforma de series web (Proyecto personal)}
\begin{itemize}[leftmargin=*, noitemsep]
    \item Plataforma web para exploración y visualización de series con base de datos \textbf{PHPMyAdmin/MySQL}.
    \item Cada serie cuenta con información, temporadas y episodios, manejados dinámicamente con \textbf{EJS} y JavaScript.
    \item Integración con servidores externos para reproducción de videos y gestión de contenidos.
\end{itemize}

\vspace{6pt}

\section*{Habilidades técnicas}

\textbf{Lenguajes:} Python, JavaScript, HTML, CSS, SQL, EJS \\
\textbf{Frameworks y librerías:} Flask, Bootstrap, CustomTkinter \\
\textbf{Bases de datos:} SQLite, MySQL (PHPMyAdmin) \\
\textbf{Herramientas:} Git, VS Code, Excel, JSON, CSV \\
\textbf{Áreas de conocimiento:} Desarrollo de aplicaciones de escritorio, diseño web, control de inventarios, automatización de tareas, gestión de datos, UX básico.

\vspace{6pt}

\section*{Competencias personales}

Responsabilidad, organización, resolución de problemas, aprendizaje autónomo, comunicación efectiva, atención al detalle, trabajo en equipo, orientación a resultados.

\end{document}
